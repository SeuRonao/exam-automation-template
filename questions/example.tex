\element{example}{
    \begin{question}{example-true-false}
        This proposition is true.
        \begin{choiceshoriz}[o] % fix order of alternatives even with shuffling enabled
            \correctchoice{True}
            \wrongchoice{False}
        \end{choiceshoriz}
        \explain{A true statement is true.}
    \end{question}
}

\element{example}{
    \begin{question}{example-one-choice}
        What is equal to one in \(\mathbb{Z}\)?
        \begin{choiceshoriz}
            \correctchoice{1}
            \wrongchoice{2}
            \wrongchoice{3}
            \wrongchoice{4}
            \wrongchoice{5}
        \end{choiceshoriz}
        \explain{There is only one \(1\) in the set of integers.}
    \end{question}
}

\element{example}{
    \begin{questionmult}{example-multiple-choice}
        Which below are the odd numbers?
        \begin{choiceshoriz}
            \correctchoice{1}
            \wrongchoice{2}
            \correctchoice{3}
            \wrongchoice{4}
            \correctchoice{5}
        \end{choiceshoriz}
        \explain{All odd numbers are divisible by \(2\).}
    \end{questionmult}
}

\element{example}{
    \begin{question}{example-lastchoices}\QuestionIndicative % this question is not taken into account for scoring
        Which color do you prefer?
        \begin{choiceshoriz}
            \correctchoice{red}
            \correctchoice{green}
            \correctchoice{blue}
            \lastchoices % these last options are always fixed even with shuffling enabled
            \correctchoice{I don't care}
            \correctchoice{I don't know}
        \end{choiceshoriz}
        \explain{There is no wrong answer.}
    \end{question}
}

\element{example}{
    \begin{question}{example-open}
        Write \(\pi\) with two decimal places.
        \AMCOpen{answer=3.14}{\wrongchoice[W]{w}\scoring{0}\wrongchoice[P]{p}\scoring{1}\correctchoice[C]{c}\scoring{2}}
        \explain{\(\pi = 3.14\).}
    \end{question}
}
