\element{example}{
    \begin{question}{example-true-false}
        Essa proposição é verdadeira.
        \begin{choiceshoriz}[o] % o para fixar a ordem das alternativas
            \correctchoice{Verdadeiro}
            \wrongchoice{Falso}
        \end{choiceshoriz}
        \explain{Uma proposição verdadeira é verdadeira.}
    \end{question}
}

\element{example}{
    \begin{question}{example-multiple}
        O que é igual à um em \(\mathbb{Z}\)?
        \begin{choiceshoriz}
            \correctchoice{1}
            \wrongchoice{2}
            \wrongchoice{3}
            \wrongchoice{4}
            \wrongchoice{5}
        \end{choiceshoriz}
        \explain{Só temos um \(1\) nos números inteiros!}
    \end{question}
}

\element{example}{
    \begin{questionmult}{example-multiple}
        O que é igual à um em \(\mathbb{Z}_2\)?
        \begin{choiceshoriz}
            \correctchoice{1}
            \wrongchoice{2}
            \correctchoice{3}
            \wrongchoice{4}
            \correctchoice{5}
        \end{choiceshoriz}
        \explain{Todos os ímpares tem resto \(1\) quando divididos por \(2\).}
    \end{questionmult}
}

\element{example}{
    \begin{question}{example-lastchoices}\QuestionIndicative % desativa a contagem de pontos dessa questão
        Qual cor?
        \begin{choiceshoriz}
            \wrongchoice{vermelho}
            \wrongchoice{azul}
            \wrongchoice{amarelo}
            \lastchoices % fixa as últimas alternativas
            \correctchoice{transparente}
            \wrongchoice{não sei}
        \end{choiceshoriz}
        \explain{Não existe resposta correta.
        Essa pergunta é só para mostrar que é possível fixar as últimas alternativas.
        }
    \end{question}
}

\element{example}{
    \begin{question}{example-open}
        Escreva \(\pi\) com duas casas decimais.
        \AMCOpen{answer=3.14}{\wrongchoice[W]{w}\scoring{0}\wrongchoice[P]{p}\scoring{1}\correctchoice[C]{c}\scoring{2}}
        \explain{O valor é \(3.14\).}
    \end{question}
}

\element{example}{
    \begin{question}{example-random-lua}
        \directlua{
            a = math.random(0, 10)
            b = math.random(0, 10)
        }
        Quanto vale \(\directlua{tex.print(a)} + \directlua{tex.print(b)}\)?
        \begin{choiceshoriz}
            \wrongchoice{\directlua{tex.print(a+b+1)}}
            \wrongchoice{\directlua{tex.print(a+b-1)}}
            \correctchoice{\directlua{tex.print(a+b)}}
            \wrongchoice{\directlua{tex.print(a+b+2)}}
            \wrongchoice{\directlua{tex.print(a+b-2)}}
        \end{choiceshoriz}
    \end{question}
}

\element{example}{
    \begin{questionmultx}{example-random-lua-open}
        \directlua{
            a = math.random(0, 100)
            r = a^0.5
        }
        Quanto vale \(\sqrt{\directlua{tex.print(a)}}\)?
        \AMCnumericChoices{\directlua{tex.print(r)}}{digits=5, decimals=2}
        \explain{O valor é \(\directlua{tex.print(r)}\).}
    \end{questionmultx}
}

\element{example}{
    \begin{questionmultx}{example-lua-import}
        \directlua{
            module = require("/home/seuronao/projects/seuronao/exam-cc0265-prob-est/utils/example.lua")
            a = math.random(-9, 9)
            b = math.random(-9, 9)
            resposta = module.mult(a, b)
        }
        Quanto vale \(\directlua{tex.print(a)} \times \directlua{tex.print(b)}\)?
        \AMCnumericChoices{\directlua{tex.print(resposta)}}{digits=2}
    \end{questionmultx}
}