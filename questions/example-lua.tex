\element{example-lua}{
	\begin{question}{example-lua-random}
		\begin{luacode}
			a = math.random(0, 10)
			b = math.random(0, 10)
		\end{luacode}
		How much is \(\directlua{tex.print(a)} + \directlua{tex.print(b)}\)?
		\begin{choiceshoriz}
			\wrongchoice{\directlua{tex.print(a+b-2)}}
			\wrongchoice{\directlua{tex.print(a+b-1)}}
			\correctchoice{\directlua{tex.print(a+b)}}
			\wrongchoice{\directlua{tex.print(a+b+1)}}
			\wrongchoice{\directlua{tex.print(a+b+2)}}
		\end{choiceshoriz}
		\explain{The value is \(\directlua{tex.print(a+b)}\).}
	\end{question}
}

\element{example-lua}{
	\begin{questionmultx}{example-lua-random-open}
		\begin{luacode}
			a = math.random(0, 100)
			r = a^0.5
		\end{luacode}
		How much is \(\sqrt{\directlua{tex.print(a)}}\)?
		\AMCnumericChoices{\directlua{tex.print(r)}}{digits=4, decimals=2}
		\explain{The value is \(\directlua{tex.print(r)}\).}
	\end{questionmultx}
}

\element{example-lua}{
	\begin{questionmultx}{example-lua-import}
		\begin{luacode}
			a = math.random(-9, 9)
			b = math.random(-9, 9)
			resposta = example_lua.mult(a, b)
		\end{luacode}
		How much is \(\directlua{tex.print(a)} \times \directlua{tex.print(b)}\)?
		\AMCnumericChoices{\directlua{tex.print(resposta)}}{digits=2}
		\explain{The value is \(\directlua{tex.print(resposta)}\).}
	\end{questionmultx}
}
