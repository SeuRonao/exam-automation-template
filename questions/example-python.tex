\element{example-python}{
	\begin{question}{example-python-random}
		\pyc{a = random.randint(0, 10)}
		\pyc{b = random.randint(0, 10)}
		How much is \(\py{a} + \py{b}\)?
		\begin{choiceshoriz}
			\wrongchoice{\py{a+b-2}}
			\wrongchoice{\py{a+b-1}}
			\correctchoice{\py{a+b}}
			\wrongchoice{\py{a+b+1}}
			\wrongchoice{\py{a+b+2}}
		\end{choiceshoriz}
		\explain{The value is \(\py{a+b}\).}
	\end{question}
}

\element{example-python}{
	\begin{questionmultx}{example-python-random-open}
		\pyc{a = random.randint(0, 100)}
		\pyc{r = a**0.5}
		How much is \(\sqrt{\py{a}}\)?
		\pys{\AMCnumericChoices{!{r}}{digits=4, decimals=2}}
		\explain{The value is \(\py{r}\).}
	\end{questionmultx}
}

\element{example-python}{
	\begin{questionmultx}{example-python-import}
		\pyc{a = random.randint(-10, 10)}
		\pyc{b = random.randint(-10, 10)}
		\pyc{result = example_python.multiply(a, b)}
		How much is \(\py{a} \times \py{b}\)?
		\pys{\AMCnumericChoices{!{result}}{digits=2}}
		\explain{The value is \(\py{result}\).}
	\end{questionmultx}
}
