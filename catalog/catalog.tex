\documentclass[a4paper]{article}
\RequirePackage{etex}
\usepackage{import}
% \usepackage[utf8]{inputenc} % disabled due to using LuaTex
% \usepackage[T1]{fontenc} % disabled due to using LuaTex
\usepackage{fontspec} % for LuaTex
\usepackage{multicol}
\usepackage{booktabs}
\usepackage{amsmath}
\usepackage{amssymb}
\usepackage{color}
\usepackage[catalog, lang=en]{automultiplechoice}
\usepackage[gobble=auto, rerun=always]{pythontex}
\usepackage{luacode}

% Exam Information
\newcommand{\ExamTitle}{\textbf{COURSE NAME\hfill Question Catalog}}
\newcommand{\ExamDate}{\today}

% Commands to set the random seed (1/2)
\begin{pycode}
	import random
	random.seed(1)
\end{pycode}

% Commands to import the lua files for the questions
\luadirect{example_lua = require "../questions/example_lua"}

% Commands to import the python files for the questions
\begin{pycode}
	import sys
	from pathlib import Path
	absolute_path_string = '../questions' # Questions folder
	sys.path.append(str(Path(absolute_path_string).resolve()))
	import example_python
\end{pycode}

% Commands to import the files with the questions
\import{../questions}{example}
\import{../questions}{example-lua}
\import{../questions}{example-python}

%%%%%%%%%%%%%%%%%%%%%%
% Document starts here
%%%%%%%%%%%%%%%%%%%%%%
\begin{document}
% Commands to set the random seed (2/2)
\AMCrandomseed{1}
\luadirect{math.randomseed(1)}
%%% beginning of the test sheet header:
\noindent\ExamTitle\hfill\textbf{\ExamDate}
%%% end of the header

The \texttt{Auto Multiple Choice} can be used alongside \texttt{Lua} or \texttt{Python} code to generate random numbers so that each realization or copy of the exam has different numbers for these questions.

There are some tipes of questions that are well suited for use with the software:

\begin{description}
	\item[Multiple Choice:] questions with only one correct answer, multiple correct answers, and even questions with no correct answers.
	\item[Range:] It is also possible to have a numeric question with a range of accepted answers.
	\item[Open Questions:] open space for the student to write a short answer or a long answer. These questions can be corrected by the teacher.
\end{description}

\section{Question Samples}

These questions do not make use of any external or internal code with \texttt{Lua} or \texttt{Python}.

\insertgroup{example}

\section{Question Samples with Lua Code}

These questions use the \texttt{Auto Multiple Choice} package with \texttt{Lua} code.
Code can be used to generate random numbers so that each realization or copy of the exam has different numbers for these questions.

It is necessary to use the \texttt{LuaTex} engine to compile the document when lua code is present.

There are some minor inconveniences, such as the need to use the commands \texttt{\textbackslash{}luadirect{}} or \texttt{\textbackslash{}directlua{}} to run lua code inside questions.

Some examples can be found below:

\insertgroup{example-lua}

While \texttt{Lua} is a very powerful language, it does not have all the libraries and packages such as a more popular language such as \texttt{Python}.

\section{Question Samples with Python Code}

These questions use the \texttt{Auto Multiple Choice} package with \texttt{Python} code.
Code can be used to generate random numbers so that each realization or copy of the exam has different numbers for these questions.

It is necessary to use the \texttt{pythontex} package to compile the document when python code is present.

There are some minor inconveniences, such as the need to use the commands \texttt{\textbackslash{}pyc{}} and \texttt{\textbackslash{}py{}} to run python code inside questions.

However due to the popularity of \texttt{Python}, it is possible to use many libraries and packages to generate questions.

\insertgroup{example-python}

\end{document}
