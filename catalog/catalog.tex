\documentclass[a4paper]{article}
\RequirePackage{etex}
\usepackage{import}
% \usepackage[utf8]{inputenc} % disabled due to using LuaTex
% \usepackage[T1]{fontenc} % disabled due to using LuaTex
\usepackage{fontspec} % for LuaTex
\usepackage{multicol}
\usepackage{booktabs}
\usepackage{amsmath}
\usepackage{amssymb}
\usepackage{color}
\usepackage[catalog, lang=en]{automultiplechoice}
\usepackage[rerun=always]{pythontex}

% Exam Information
\newcommand{\ExamTitle}{\textbf{COURSE NAME\hfill Question Catalog}}
\newcommand{\ExamDate}{\today}

% Commands to set the random seed (1/2)
\begin{pycode}
import random
random.seed(1)
\end{pycode}

% Commands to import the lua files for the questions
\directlua{
example_lua = require "../questions/example_lua"
}

% Commands to import the python files for the questions
\begin{pycode}
import sys
from pathlib import Path
absolute_path_string = '../questions' # Questions folder
sys.path.append(str(Path(absolute_path_string).resolve()))
import example_python
\end{pycode}

% Commands to import the files with the questions
\import{../questions}{example}
\import{../questions}{example-lua}
\import{../questions}{example-python}

%%%%%%%%%%%%%%%%%%%%%%
% Document starts here
%%%%%%%%%%%%%%%%%%%%%%
\begin{document}
% Commands to set the random seed (2/2)
\AMCrandomseed{1}
\directlua{math.randomseed(1)}
%%% beginning of the test sheet header:
\noindent{\ExamTitle}
\vspace*{.5cm}
\begin{minipage}{.4\linewidth}
    \centering\large\textbf{\ExamDate}
\end{minipage}
%%% end of the header

These questions use the \texttt{Auto Multiple Choice} package without any additional code.
Some examples can be found below:

\begin{description}
    \item[Multiple Choice:] questions with only one correct answer, multiple correct answers, and even questions with no correct answers.
    \item[Open Questions:] open space for the student to write a short answer or a long answer.
    \item[Range:] It is also possible to have a numeric question with a range of accepted answers.
\end{description}

\section{Question Samples}

\insertgroup{example}

\section{Question Samples with Lua Code}

These questions use the \texttt{Auto Multiple Choice} package with \texttt{Lua} code.
Code can be used to generate random numbers so that each realization or copy of the exam has different numbers for these questions.

It is necessary to use the \texttt{LuaTex} engine to compile the document when lua code is present.
Some examples can be found below:

\insertgroup{example-lua}

While \texttt{Lua} is a very powerful language, it does not have all the libraries and packages such as a more popular language such as \texttt{Python}.

\section{Question Samples with Python Code}

These questions use the \texttt{Auto Multiple Choice} package with \texttt{Python} code.
Code can be used to generate random numbers so that each realization or copy of the exam has different numbers for these questions.

It is necessary to use the \texttt{pythontex} package to compile the document when python code is present.

There are some minor inconveniences, such as the need to use the commands \texttt{\textbackslash pyc{}} and \texttt{\textbackslash py{}} to run python code inside questions.

However due to the popularity of \texttt{Python}, it is possible to use many libraries and packages to generate questions.

\insertgroup{example-python}

\end{document}
