\documentclass[a4paper]{article}
\RequirePackage{etex}
\usepackage{import}
% \usepackage[utf8]{inputenc} % disabled due to using LuaTex
% \usepackage[T1]{fontenc} % disabled due to using LuaTex
\usepackage{fontspec} % for LuaTex
\usepackage{multicol}
\usepackage{booktabs}
\usepackage{amsmath}
\usepackage{amssymb}
\usepackage{color}
\usepackage[catalog, lang=en]{automultiplechoice}
\usepackage{pythontex}

% Exam Information
\newcommand{\ExamTitle}{\textbf{COURSE NAME\hfill Question Catalog}}
\newcommand{\ExamDate}{\today}

% Commands to set the random seed (1/2)
\begin{pycode}
import random
random.seed(1)
\end{pycode}

% Commands to import the files with the questions
\import{../questions}{example}
\import{../questions}{example-lua}
\import{../questions}{example-python}

%%%%%%%%%%%%%%%%%%%%%%
% Document starts here
%%%%%%%%%%%%%%%%%%%%%%
\begin{document}
% Commands to set the random seed (2/2)
\AMCrandomseed{1}
%%% beginning of the test sheet header:
\noindent{\ExamTitle}
\vspace*{.5cm}
\begin{minipage}{.4\linewidth}
    \centering\large\textbf{\ExamDate}
\end{minipage}
%%% end of the header

\section{Question Samples}
\insertgroup{example}
\section{Question Samples with Lua Code}
\insertgroup{example-lua}
\section{Question Samples with Python Code}
\insertgroup{example-python}
\end{document}
